%!TEX root = ./CV.tex

\begin{rubric}{Research Experience}
    % should I call it ``summer long-distance research programme'' technically?
    \entry*[2022] \textbf{Summer Internship} \hfill \emph{Guides: Piyush Srivastava and Hariharan Narayanan $\mid$ TIFR, Mumbai}

        $\bullet$ Analyzed a novel multiscale Markov chain on convex bodies that mixes rapidly from a cold start

        $\bullet$ Proved that the coordinate hit-and-run Markov chain mixes rapidly from a cold start

        % $\bullet$ Wrote a paper on the above (\href{https://arxiv.org/abs/2211.04439}{arXiv:2211.04439}), and submitted it to STOC 2023 \phantom{\cite{multiscale}}

    \entry*[\phantom{.}2022] \textbf{B.Tech. Project} \hfill \emph{Guides: Prof. Niranjan Balachandran and Prof. Rohit Gurjar $\mid$ IIT Bombay}

        $\bullet$ Working towards proving Bagchi's conjecture, a problem in combinatorial geometry

        $\bullet$ Studied various results in the analysis of boolean functions, including the KKL Theorem and a result on independent sets in graph products due to Dinur, Friedgut, and Regev

        $\bullet$ Covered various general methods to solve combinatorial problems, also preparing a report on all the topics and papers studied, which can be found \href{https://amitrajaraman.github.io/research/pls/btp1-report.pdf}{here}.
    
    \entry*[2021] \textbf{Summer Internship} \hfill \emph{Guide: Navin Goyal $\mid$ Microsoft Research, Bengaluru}
        
        $\bullet$ Worked towards proving the KLS Conjecture and Hyperplane Slicing Conjecture, elusive problems in high-dimensional geometry, using the localization and stochastic localization methods

        $\bullet$ Prepared a report on the topics studied, covering several topics in asymptotic convex geometry from scratch, which can be found \href{https://amitrajaraman.github.io/notes/convex-geometry/main.pdf}{here}

\end{rubric}